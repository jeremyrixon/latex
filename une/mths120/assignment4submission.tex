\documentclass[12pt,a4paper]{report}
\usepackage[margin=1in]{geometry}
\usepackage{amssymb}
\usepackage{amsmath}
\usepackage{siunitx}
\setlength\parindent{0pt}
\linespread{2.25}
\begin{document}


\textbf{MTHS120 Assessment 4}

\textbf{Question 1.}

\textbf{1a.}

Given:
\begin{align}
V &= IR \\
V &= 20\text{ volts} \\
I &= 2\text{ amps}
\end{align}

Solving for R:
\begin{equation}
R = 10\text{ ohms} \\
\end{equation}

\textbf{1b.}

Using the product rule:
\begin{equation}
\frac{dV}{dt}=R\frac{dI}{dt} + I\frac{dR}{dt}
\end{equation}

Given:
\begin{align}
\frac{dV}{dt} &= 0\text{ volts/s} \\
\frac{dR}{dt} &= 0.4 \text{ ohms/s}
\end{align}

Substituting and solving:
\begingroup
\addtolength{\jot}{0.5em}
\begin{align}
0 &=10\frac{dI}{dt} + 0.8 \\
10\frac{dI}{dt} &= -0.8 \\
\frac{dI}{dt} &= -0.08 \text{ amps/s}
\end{align}
\endgroup

\newpage

\textbf{Question 2.}

\textbf{2a.}

Given:
\begingroup
\addtolength{\jot}{0.5em}
\begin{align}
PV &= nRT \\
R &\approx 8.3145 \text{ kPaL/Kmol} \\
n &= 1\text{ mole} \\
V &= 25\text{ litres} \\
\frac{dT}{dt} &= 3.5\text{ kelvin / min} 
\end{align}
\endgroup

If V, n and R are constant:
\begingroup
\addtolength{\jot}{0.5em}
\begin{align}
V\frac{dP}{dt} &= nR\frac{dT}{dt}
\end{align}
\endgroup

Substituting and solving for dP/dt:
\begingroup
\addtolength{\jot}{0.5em}
\begin{align}
\frac{dP}{dt} &\approx \frac{8.3145 \times 3.5}{25} \\
\frac{dP}{dt} &\approx 1.16403 \text{ kPa/min}
\end{align}
\endgroup

\textbf{2b.}

Given:
\begingroup
\addtolength{\jot}{0.5em}
\begin{align}
PV &= nRT \\
V &= 20\text{ litres}  \\
n &= 1.0  \\
R &\approx 8.3145 \text{ kPaL/Kmol} \\
T &= 300\text{ K} \\
\end{align}
\endgroup

We can solve for P:
\begingroup
\addtolength{\jot}{0.5em}
\begin{align}
P &\approx  \frac{1.0 \times 8.3145 \times 300}{20} \\
P &\approx  124.7175 \text{ kPa}
\end{align}
\endgroup

Given also:
\begingroup
\addtolength{\jot}{0.5em}
\begin{align}
\frac{dT}{dt} &= 0.0  \\
\frac{dV}{dt} &= -2.0
\end{align}
\endgroup

Using the product rule:
\begin{equation}
V\frac{dP}{dt} + P \frac{dV}{dt} = nR\frac{dT}{dt}
\end{equation}

Substituting and solving for dP/dt:
\begingroup
\addtolength{\jot}{0.5em}
\begin{align}
20 \frac{dP}{dt} + 124.7175 \times -2.0 & \approx 0 \\
20 \frac{dP}{dt} & \approx 249.435 \\
\frac{dP}{dt} & \approx 12.47175 \text{ kPa/s}
\end{align}
\endgroup

\textbf{Question 3.}

If \(R\) is the radius and \(h\) is the height then, taking \(\varepsilon\) to be the thickness of the sides and bottom (so the thickness of the top will be \( 3\varepsilon\)), the volume of material is given by (thanks Adam!):

\begingroup
\addtolength{\jot}{0.5em}
\begin{align}
A &= 2\pi Rh \varepsilon + 4\pi R^2 \varepsilon
\end{align}
\endgroup

Divide both sides by \(\varepsilon\):

\begingroup
\addtolength{\jot}{0.5em}
\begin{align}
\frac{A}{\varepsilon} &= 2\pi Rh + 4\pi R^2
\end{align}
\endgroup


The volume \(V\) is:
\begingroup
\addtolength{\jot}{0.5em}
\begin{align}
V = 400 = \pi R^2 h
\end{align}
\endgroup

Expressing \(h\) in terms of \(R\):
\begingroup
\addtolength{\jot}{0.5em}
\begin{align}
h &= \frac{400}{\pi R^2}
\end{align}
\endgroup

Substituting and simplifying:
\begingroup
\addtolength{\jot}{0.5em}
\begin{align}
\frac{A}{\varepsilon} &= \frac{800\pi R}{\pi R^2} + 4\pi R^2 \\
\frac{A}{\varepsilon} &= \frac{800}{R} + 4\pi R^2
\end{align}
\endgroup

Finding \(\frac{d \frac{A}{\varepsilon} } {dR} \) :
\begingroup
\addtolength{\jot}{0.5em}
\begin{align}
\frac{d \frac{A}{\varepsilon} } {dR} &= -800 R ^{-2} + 8 \pi R
\end{align}
\endgroup

Solving for  \(\frac{d \frac{A}{\varepsilon} } {dR} = 0 \) :
\begingroup
\addtolength{\jot}{0.5em}
\begin{align}
0 &= -800 R ^{-2} + 8 \pi R \\
800R^{-2} &= 8 \pi R \\
100 &= \pi R^3 \\
\frac{100}{\pi} &= R^3 \\
R &= \sqrt[3] { \frac{100}{\pi}  } \\
R &\approx 3.169202884 \text{ cm}
\end{align}
\endgroup

The limit of A as R goes to 0 is infinity and the limit of A as R goes to infinity is also infinity so A is minimised by \(R= \sqrt[3] { \frac{100}{\pi}  }\) cm.


\textbf{Question 4.}

\textbf{4a.}

The total amount of work, \(w\), expended to break a whelk shell is the number of drops, \(D(h)\), multiplied by the work per drop, \(W(h)\):
\begingroup
\addtolength{\jot}{0.5em}
\begin{align}
w &= D(h) W(h)\end{align}
\endgroup

Substituting and simplifying:
\begingroup
\addtolength{\jot}{0.5em}
\begin{align}
w &= (1 + \frac{20.4}{h-0.84}) Kh \\
\frac{w}{K} &= h + \frac{20.4h}{h-0.84} 
\end{align}
\endgroup

Finding \(\frac{d \frac{w}{K} } {dh} \) (applying the quotient rule):
\begingroup
\addtolength{\jot}{0.5em}
\begin{align}
\frac{d \frac{w}{K} } {dh} &=  1 + \frac{(20.4)(h-0.84) - (20.4h)}{(h-0.84)^2} \\
\frac{d \frac{w}{K} } {dh} &=  1 + \frac{20.4h-17.136 - 20.4h}{(h-0.84)^2} \\
\frac{d \frac{w}{K} } {dh} &=  1 + \frac{-17.136}{(h-0.84)^2} 
\end{align}
\endgroup

Solving for  \(\frac{d \frac{w}{K} } {dh} = 0 \) :
\begingroup
\addtolength{\jot}{0.5em}
\begin{align}
0 &=  1 + \frac{-17.136}{(h-0.84)^2}  \\
1 &=  \frac{17.136}{(h-0.84)^2} \\
 (h-0.84)^2 &= 17.136 \\
h-0.84 &= 4.1396   \\
h &= 4.9796 \text{ metres}
\end{align}
\endgroup


The limit of \(w\) as \(h\) approaches \(0.84\) is infinity and the limit of \(w\) as \(h\) approaches infinity is also infinity so \(w\) is minimised by \( h = 4.9796 \) metres.


\textbf{4b.}

The average number of drops is therefore:
\begingroup
\addtolength{\jot}{0.5em}
\begin{align}
D(4.9796) & = 1 + \frac{20.4}{4.9796-0.84} \\
D(4.9796) &= 5.9280 \text{ drops}
\end{align}
\endgroup

\end{document}
