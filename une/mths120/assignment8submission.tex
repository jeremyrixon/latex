\documentclass[12pt,a4paper]{report}
\usepackage[margin=1in]{geometry}
\usepackage{amssymb}
\usepackage{amsmath}
\usepackage{siunitx}
\setlength\parindent{0pt}
\linespread{1.5}
\begin{document}

\textbf{MTHS120 Assessment 8}

\textbf{Question 1.}

We have the following system of linear equations:
\begin{align*}
 1/5c + 3/20l + 1/10z &= 300 \\
  10c +    5l +    4z &= 13900 \\
   3c +    2l +  5/2z  &= 5325
\end{align*} 

Represented in matrix form and using Gauss-Jordan elimination:
 \[
 \left(\begin{array}{rrr|r}
   1/5 & 3/20 & 1/10 & 300  \\
   10  & 5    & 4    & 13900  \\
   3   & 2    & 5/2  & 5325  \\ 
   \end{array} \right)
\]

\( R1 \rightarrow R1 \times 5 \):
 \[
 \left(\begin{array}{rrr|r}
   1   & 3/4  & 1/2  & 1500  \\
   10  & 5    & 4    & 13900  \\
   3   & 2    & 5/2  & 5325  \\ 
   \end{array} \right)
\]

\( R1 \rightarrow R2 - R1 \times 10 \):
 \[
 \left(\begin{array}{rrr|r}
   1   & 3/4  & 1/2  & 1500  \\
   0   & -5/2 & -1   & -1100  \\
   3   & 2    & 5/2  & 5325  \\ 
   \end{array} \right)
\]

\( R3 \rightarrow R3 - R1 \times 3 \):
 \[
 \left(\begin{array}{rrr|r}
   1   & 3/4  & 1/2  & 1500  \\
   0   & -5/2 & -1   & -1100  \\
   0   & -1/4 & 1    & 825  \\ 
   \end{array} \right)
\]

\( R2 \rightarrow R2 \times -2/5 \):
 \[
 \left(\begin{array}{rrr|r}
   1   & 3/4  & 1/2  & 1500  \\
   0   & 1    & 2/5  & 440  \\
   0   & -1/4 & 1    & 825  \\ 
   \end{array} \right)
\]

\( R3 \rightarrow R3 + R2 \times 1/4 \):
 \[
 \left(\begin{array}{rrr|r}
   1   & 3/4  & 1/2  & 1500  \\
   0   & 1    & 2/5  & 440  \\
   0   & 0    & 11/10& 935  \\ 
   \end{array} \right)
\]

\( R1 \rightarrow R1 - R2 \times 3/4 \):
 \[
 \left(\begin{array}{rrr|r}
   1   & 0    & 1/5  & 1170  \\
   0   & 1    & 2/5  & 440  \\
   0   & 0    & 11/10& 935  \\ 
   \end{array} \right)
\]

\( R3 \rightarrow R3 \times 10/11 \):
 \[
 \left(\begin{array}{rrr|r}
   1   & 0    & 1/5  & 1170  \\
   0   & 1    & 2/5  & 440  \\
   0   & 0    & 1    & 850  \\ 
   \end{array} \right)
\]

\( R1 \rightarrow R1 - R3 \times 1/5 \):
 \[
 \left(\begin{array}{rrr|r}
   1   & 0    & 0    & 1000  \\
   0   & 1    & 2/5  & 440  \\
   0   & 0    & 1    & 850  \\ 
   \end{array} \right)
\]

\( R2 \rightarrow R2 - R3 \times 2/5 \):
 \[
 \left(\begin{array}{rrr|r}
   1   & 0    & 0    & 1000  \\
   0   & 1    & 0    & 100  \\
   0   & 0    & 1    & 850  \\ 
   \end{array} \right)
\]

So we can read off the values:
\begin{align*}
 copper &= 1000\text{ tonnes} \\
 lead   &= 100\text{ tonnes} \\
 zinc   &= 850\text{ tonnes}
\end{align*} 


\textbf{Question 2.}

We can write the chemical equation as:
\begin{align*}
 x_1\text{Fe} + x_2\text{O\textsubscript{2}} &= x_3\text{Fe\textsubscript{2}O\textsubscript{3}}
\end{align*} 

And rewrite in terms of balancing Fe and O on each side:
\begin{align*}
x_1 - 2x_3 &= 0 \quad \text{ (for Fe)}\\
2x_2 - 3x_3 &= 0 \quad \text{ (for O)}
\end{align*} 

Represented in matrix form and using Gauss-Jordan elimination:
 \[
 \left(\begin{array}{rrr}
   1 & 0 & -2 \\
   0 & 2 & -3 \\
   \end{array} \right)
\]

\( R2 \rightarrow R2/2 \):
 \[
 \left(\begin{array}{rrr}
   1 & 0 & -2 \\
   0 & 1 & -3/2 \\
   \end{array} \right)
\]

So we have:
\begin{align*}
x_1 - 2x_3 &= 0 \\
x_2 - 3/2x_3 &= 0
\end{align*} 

Parametrically in terms of \(t\):
\begin{align*}
\begin{bmatrix}x_1\\x_2\\x_3\end{bmatrix} &= t \begin{bmatrix}2\\3/2\\1\end{bmatrix}
\end{align*} 

Setting \(t\) to \(2\):
\begin{align*}
\begin{bmatrix}x_1\\x_2\\x_3\end{bmatrix} &= \begin{bmatrix}4\\3\\2\end{bmatrix}
\end{align*} 

Gives:
\begin{align*}
4\text{Fe} + 3\text{O\textsubscript{2}} &= 2\text{Fe\textsubscript{2}O\textsubscript{3}}
\end{align*} 

\textbf{Question 3.}

We can write the chemical equation as:
\begin{align*}
 x_1\text{Fe\textsubscript{2}O\textsubscript{3}} + x_2\text{C} &= x_3\text{Fe} + x_4\text{CO\textsubscript{2}}
\end{align*} 

And rewrite in terms of balancing Fe, O and C on each side:
\begin{align*}
2x_1 - x_3 &= 0 \quad \text{ (for Fe)}\\
3x_1 - 2x_4 &= 0 \quad \text{ (for O)} \\
x_2 - x_4 &= 0 \quad \text{ (for C)}
\end{align*} 

Represented in matrix form and using Gauss-Jordan elimination:
 \[
 \left(\begin{array}{rrrr}
   2 & 0 & -1 & 0 \\
   3 & 0 & 0  & -2 \\
   0 & 1 & 0  & -1 \\
   \end{array} \right)
\]

\( R1 \rightarrow R1/2 \):
 \[
 \left(\begin{array}{rrrr}
   1 & 0 & -1/2 & 0 \\
   3 & 0 & 0  & -2 \\
   0 & 1 & 0  & -1 \\
   \end{array} \right)
\]

Swap R2 and R3:
 \[
 \left(\begin{array}{rrrr}
   1 & 0 & -1/2 & 0 \\
   0 & 1 & 0  & -1 \\
   3 & 0 & 0  & -2 \\
   \end{array} \right)
\]

\( R3 \rightarrow R3 - R1 \times 3 \):
 \[
 \left(\begin{array}{rrrr}
   1 & 0 & -1/2 & 0 \\
   0 & 1 & 0  & -1 \\
   0 & 0 & 3/2 & -2 \\
   \end{array} \right)
\]

\( R3 \rightarrow R3 \times 2/3 \):
 \[
 \left(\begin{array}{rrrr}
   1 & 0 & -1/2 & 0 \\
   0 & 1 & 0  & -1 \\
   0 & 0 & 1 & -4/3 \\
   \end{array} \right)
\]

\( R1 \rightarrow R1 + R3 \times 1/2 \):
 \[
 \left(\begin{array}{rrrr}
   1 & 0 & 0 & -2/3 \\
   0 & 1 & 0  & -1 \\
   0 & 0 & 1 & -4/3 \\
   \end{array} \right)
\]

So we have:
\begin{align*}
x_1 - 2/3x_4 &= 0 \\
x_2 -    x_4 &= 0 \\
x_3 -    4/3x_4 &= 0
\end{align*} 

Parametrically in terms of \(t\):
\begin{align*}
\begin{bmatrix}x_1\\x_2\\x_3\\x_4\end{bmatrix} &= t \begin{bmatrix}2/3\\1\\4/3\\1\end{bmatrix}
\end{align*} 

Setting \(t\) to \(3\):
\begin{align*}
\begin{bmatrix}x_1\\x_2\\x_3\\x_4\end{bmatrix} &= \begin{bmatrix}2\\3\\4\\3\end{bmatrix}
\end{align*} 

Gives:
\begin{align*}
 2\text{Fe\textsubscript{2}O\textsubscript{3}} + 3\text{C} &= 4\text{Fe} + 3\text{CO\textsubscript{2}}
\end{align*} 

\textbf{Question 4.}

Plugging the three given points into the quadratic polynomial gives us:
\begin{align*}
  a +  b + c &= 4  \\
 4a + 2b + c &= 2  \\
 9a + 3b + c &= -2 
\end{align*} 


Represented in matrix form and using Gauss-Jordan elimination:
 \[
 \left(\begin{array}{rrr|r}
   1 & 1 & 1 & 4 \\
   4 & 2 & 1 & 2  \\
   9 & 3 & 1 & -2  \\ 
   \end{array} \right)
\]

\( R2 \rightarrow R2 - 4 \times R1\):
 \[
 \left(\begin{array}{rrr|r}
   1 & 1  & 1  & 4 \\
   0 & -2 & -3 & -14  \\
   9 & 3  & 1  & -2  \\ 
   \end{array} \right)
\]

\( R3 \rightarrow R3 - 9 \times R1 \):
 \[
 \left(\begin{array}{rrr|r}
   1 & 1  & 1  & 4 \\
   0 & -2 & -3 & -14  \\
   0 & -6 & -8 & -38  \\ 
   \end{array} \right)
\]

\( R2 \rightarrow R2 \times -1/2\):
 \[
 \left(\begin{array}{rrr|r}
   1 & 1  & 1  & 4 \\
   0 & 1 & 1.5 & 7  \\
   0 & -6 & -8 & -38  \\ 
   \end{array} \right)
\]

\( R3 \rightarrow R3 + 6 \times R2\):
 \[
 \left(\begin{array}{rrr|r}
   1 & 1  & 1  & 4 \\
   0 & 1 & 1.5 & 7  \\
   0 & 0 & 1 & 4  \\ 
   \end{array} \right)
\]

\( R1 \rightarrow R1 - R2\):
 \[
 \left(\begin{array}{rrr|r}
   1 & 0 & -0.5 & -3 \\
   0 & 1 & 1.5 & 7  \\
   0 & 0 & 1 & 4  \\ 
   \end{array} \right)
\]

\( R1 \rightarrow R1 + R3/2\):
 \[
 \left(\begin{array}{rrr|r}
   1 & 0 & 0 & -1 \\
   0 & 1 & 1.5 & 7  \\
   0 & 0 & 1 & 4  \\ 
   \end{array} \right)
\]

\( R2 \rightarrow R2 - R3/2\):
 \[
 \left(\begin{array}{rrr|r}
   1 & 0 & 0 & -1 \\
   0 & 1 & 0 & 1  \\
   0 & 0 & 1 & 4  \\ 
   \end{array} \right)
\]

So:
\begin{align*}
a &= -1 \\
b &= 1 \\
c &= 4
\end{align*}

So the quadratic polynomial is:
\begin{align*}
y = -x^2 + x + 4
\end{align*}


\end{document}

